\section{GlaisterShallit}
%begin detailed
\begin{frame}{Глейстер-Шаллит}
    Пусть $L \in \Sigma \star$ это регулярный язык и существует множество пар $P = \{(x_i, w_i) | 1 \leqslant n\}$ таких, что:
    \begin{itemize}
        \item $x_i w_i \in L$ для $1 \leqslant i \leqslant n$
        \item $x_i w_i \notin L$ для $1 \leqslant i, j \leqslant n$, и $i \neq j$
    \end{itemize}
    Тогда любой НКА, считывающий $L$, имеет не менее n состояний.
    \begin{block}{\bf Доказательство}
        Пусть $M = (Q, \Sigma, \delta, q_0, F)$ любой НКА считывающий $L$. Рассмотрим множество состояний $S = \delta(q_0, x_i)$. Так как ${x_i}{w_i} \in L$ должно существовать состояние $p_i \in S$, такое что $\delta(p_i, w_i) \cap F$ непусто. То есть существует состояние $r_i \in F$, где $r_i \in \delta(p_i, w_i)$. Если $p_i \in \delta(q_0, x_j)$, то $r_i \in \delta(p_i, w_i) \subseteq \delta(q_0, {x_j}{w_i})$, таким образом ${x_j}{w_i} \in L$ — противоречие, следовательно $p_i \notin \delta(q_0, x_j)$, для все $i \neq j$. Отсюда следует, что каждое множество $\delta(q_0, x_i)$, содержит состояние $p_i$, которое не содержится в любом другом множестве $\delta(q_0, x_j), i \neq j$. Следовательно, M имеет не менее n состояний.
    \end{block}
\end{frame}
\begin{frame}{Усиление Глейстера-Шаллита}
    Пусть, есть $n$ строк $w_i$ таких, что каждой $w_i$ можно поставить в соответствие такой суффикс $v_i$, что $w_i v_i$ входит в L, а $w_j v_i$ для всех $j<i$ в L не входит.

    Рассмотрим $w_n$. Переходы по $w_n$ из стартового состояния приведут в множество $(q_1, ... ,q_m)$ состояний НКА, причём как минимум из одного из них можно попасть в финальное по слову $v_n$. Выделим это состояние и назовём его $q_n$.
    
    Пусть некоторый $w_i$ (где $i<n$) переводит в $q_n$ из стартового. Это приводит к противоречию.
    
    Рассмотрим $w_{n-1}$. Для него тоже есть состояние, которое приводит по $v_{n-1}$ в финальное. Оно не совпадает с $q_n$, т.к. иначе бы $w_{n-1} v_n$ вошло в язык L, а это не так. И в это состояние не могут приводить никакие $w_i$ с меньшими индексами. Таким образом НКА имеет не менее n состояний.
\end{frame}
\begin{frame}{Усиление Глейстера-Шаллита}
    Сокращающий перебор критерий: при сортировке префиксов "по усилению", т.е. когда более нижний обязательно сочетаются с чем-то, с чем не сочетаются никакие более верхние, для добавления в класс эквивалентности достаточно проверки, на то что в столбце матрицы единственная единица. Такие столбцы сразу показывают различающий суффикс, и можно забирать соответствующую строку в очередной класс эквивалентности и вычёркивать. Это даёт переход от единичной матрицы к нижнетреугольной, что существенно улучшает метод.
\end{frame}
%end detailed
\begin{frame}{Вычисление $\GlaisterShallit\TypeIs\NFATYPE\to\IntTYPE$}
	Автомат:

	%template_oldautomaton

	Таблица:

	%template_table

	Результат:
	%template_result

	%template_cach

\end{frame}
