\section{Minimize}
%begin detailed
\begin{frame}{Минимизация DFA}
    \begin{itemize}
        \item Построим таблицу всех двухэлементных множеств $\{q_i, q_j\}, q_i, q_j \in Q$
        \item Пометим все множества $\{q_i, q_j\}$ такие, что одно из $q_i, q_j$ из $F$, а второе нет.
        \item Пометим все множества $\{q_i, q_j\}$ такие, что $\exists a({q_i}\overset{a}{\rar}{q_1}' \: \& \: {q_j}\overset{a}{\rar}{q_2}' \: \& \: \{{q_1}', {q_2}'\}$ — помеченная пара). Продолжаем помечать, пока не будет появляться новых помеченных пар.
    \end{itemize}
    Пары, оставшиеся непомеченными, можно объединить.
\end{frame}
\begin{frame}{Минимизация NFA}
    \begin{block}{Минимизация по Брзозовски}
        $det(reverse(det(reverse(A))))$ является минимальным ДКА для любого НКА $A$.
    \end{block}
    Многие алгоритмы для порождения малых (не минимальных) NFA являются комбинациями нескольких операций:
    \begin{itemize}
        \item Обращение автомата
        \item Детерминизация
        \item Удаление $\epsilon$-правил
        \item Минимизация
        \item Разметка
    \end{itemize}
\end{frame}
%end detailed
\begin{frame}{Преобразование $\Minimize\TypeIs\NFATYPE\to\DFATYPE$}
	Автомат до преобразования:

	%template_oldautomaton

	%template_to_determ

	%template_detautomaton

	Автомат после преобразования:%template_trap

	%template_result

	Классы эквивалентности:

	%template_equivclasses

	%cach:

	%template_cach

\end{frame}
